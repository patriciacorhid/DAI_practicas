\documentclass{article}
\usepackage[left=1.8cm,right=3cm,top=2cm,bottom=2cm]{geometry} % page
% settings
\usepackage{multicol} 
\usepackage{amsmath} % provides many mathematical environments & tools
\usepackage{dsfont}
\usepackage{upgreek}
\usepackage[spanish]{babel}
\usepackage[doument]{ragged2e}

% Images
\usepackage{graphicx}
\usepackage{float}
\usepackage{subfigure} % subfiguras
\usepackage{caption}
\captionsetup[table]{labelformat=empty}
\captionsetup[figure]{labelformat=empty}

% Code
\usepackage{tikz}
\usetikzlibrary{automata,positioning}
\usepackage{pgfplots}
\usepackage{color}

\usepackage{listings}
\usepackage{xcolor}
\definecolor{gray}{rgb}{0.5,0.5,0.5}
\newcommand{\n}[1]{{\color{gray}#1}}
\lstset{numbers=left,numberstyle=\small\color{gray}}

\usepackage[bookmarks=true,
            bookmarksnumbered=false, % true means bookmarks in 
                                     % left window are numbered
            bookmarksopen=false,     % true means only level 1
                                     % are displayed.
            colorlinks=true,
            allcolors=blue,
            urlcolor=cyan]{hyperref}

\selectlanguage{spanish}
\usepackage[utf8]{inputenc}
\setlength{\parindent}{0mm}

\begin{document}

\title{Práctica 10}
\author{Patricia Córdoba Hidalgo}
\date{}
\maketitle

\section{Objetivos}

\section{Alternativas}

\section{Problemas y sus soluciones}

Para calcular el año máximo y mínimo usamos la función sort como nos indica la documentación:
https://docs.mongodb.com/manual/reference/method/cursor.sort/

La línea de código de la función gráficas:
max_year = db.video_movies.find().sort({'year':-1})[0]['year']

producía el siguiente error:
if no direction is specified, key_or_list must be an instance of list

Tras consultar el siguiente enlace:
https://stackoverflow.com/questions/10242149/using-sort-with-pymongo

descubrimos que la manera correcta es:
max_year = db.video_movies.find().sort([('year',-1)])[0]['year']

\section{Proceso}

(No he programado nada aún)

Añadimos esta línea en la cabecera del html base:
<script src="https://code.highcharts.com/highcharts.js"></script>

Para crear el gráfico, seguir los pasos de:
https://www.highcharts.com/docs/getting-started/your-first-chart

Para modificar gráfico:
https://www.highcharts.com/docs/getting-started/how-to-set-options
https://jsfiddle.net/gh/get/library/pure/highcharts/highcharts/tree/master/samples/highcharts/members/setoptions

--------------------------------

Añadimos esta línea en la cabecera de p4.html:
<script src="https://code.highcharts.com/highcharts.js"></script>

Creamos graficas.html con el div con id=grafica1



\end{document}